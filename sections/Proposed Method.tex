\documentclass[../main.tex]{subfiles}
\begin{document}
\subsection{Investment targets}
what are DJI 30s

what is DJI

what is IXIC

what is NYA

Why choose these investment targets

\subsection{Technical Indicators}
What are technical indicators
% Technical indicators are the rule of thumb or pattern-based signals produced mathematically by the stock price or volume. The fundatioin of technical indicators is the historical price of the stocks. It is belived that the history will repeated itself as the time extends. In other words, patterns of the market behavior . By analyzing the historical data, technical analysis use indicators to determine market entry and exit points.

\subsubsection{Simple Moving Average (SMA)}
What is Simple Moving Average

Characteristics of SMA

SMA formula

How to use SMA


\subsubsection{Relative Strength Index (RSI)}
What is RSI

Characteristics of RSI

RSI formula

How to use RSI

% Relative Strength Index (RSI) was developed by J. Welles Wilder, Jr. [3] in 1978. This index is a widely used technical indicator of the financial market for measuring the magnitude of recent price changes. RSI regards a rising stock price as strength from buyers, a falling price as strength from sellers, and the closing price is the outcome of the relative strength of buyers and sellers. Here are the formulae for calculating RSI. The calculating process can be divided into two steps. For step one (as shown in formulae \ref{step_1}), the average gain and loss is the average of rice and drop respectively during the look-back period. As for step two, with the RSI in step one, we can recursively calculate the next RSI using formula 2, where $N$ is a parameter, representing a look-back period. The formulae for calculating RSI are as follows.

% \begin{equation}\label{step_1}\scalebox{1}
%     {$RSI_{step\ one}=100-\left[\dfrac{100}{1+\dfrac{Average\ gain}{Average\ loss}}\right]$}
%     \vspace{1em}
% \end{equation}

% \begin{equation}\label{step_2}\scalebox{1}
%     {$RSI_{step\ two}=100-\left[\dfrac{100}{1+\dfrac{Previous\ Average\ Gain\times(N-1)+Current\ Gain}{-(Previous\ Average\ Loss)\times(N-1)+Current\ Loss}}\right]$}
%     \vspace{1em}
% \end{equation}

% \begin{figure}

% \end{figure}
\subsection{Internal Rate of Return Normalization}
How to evaluate the performance of our method
\subsection{Sliding Windows}
Old sliding sindows

New sliding windows

What can new Sliding achieve
\subsection{Global-best guided Quantum-inspired Tabu Search Algorithm with Not-gate (GNQTS)}
What is GNQTS and its evolution

Flow chart of GNQTS

Pseudo code of GNQTS

Explain each step of GNQTS

\end{document}