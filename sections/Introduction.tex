\documentclass[../main.tex]{subfiles}
\begin{document}

% A literature review of technical analysis on stock markets \textsc{[]}

% \textbf{Why Invest}
People didn't like to invest in the past because they didn't familiar with investment. What they do with their income is to deposit it in the bank, which seems neither a good nor a bad choice at the moment. As the time goes by, the unseen problem grows day by day, that is the inflation. As people realize the problem with depositing money in the bank, more and more people are willingly to invest to gain their wealth other than keeping the money in the bank. Hence, investment becomes a important issue in today's society.

\bigbreak

% \textbf{How do most of people invest}
There are varieties of investment to choose from, such as, stocks, bonds, options, real estate, and futures. One of a popular investment target is stocks. The advantages of investing in stocks are transparency, liquidility, and low requirement to invest.

\bigbreak

The U.S. stock market is the target in this research. It is the largest economy in the world which has a great impact on the world financial system. Any oscillation in the U.S. stock market will reflect on every stock market all over the globe. Also there are a lot of companies and various types of companies from differenet country in the U.S. stock markets. Hence, the diversity and market capitalization make it a good choise of investing in the U.S. market.

\bigbreak

The basic idea of investing in stocks is to buy at low price and sell at high price. The idea is simple, but hard to achieve without any analysis of stock market. So, most of the people use two types of method to analyze the company that investor is interested in or the stock market. There are two method for investors to analyze a stock is worth buying or not, one is fundamental analysis, the other one is technical analysis.

\bigbreak

% \textbf{Fundamental analysis}
Fundamental analysis in the stock market is a method of evaluating the operating conditions of a company, such as governance, revenues, earnings, future growth, return on equity, profit margins, and other data. All of this data is available in a company's financial statements. These factors refer to estimating a company's underlying value and potential for future growth. Buy and sell decisions are then made based on whether the investors believe that those factors will have positive or negative impact on the stock price. If an analyst calculates that the value of the stock is expected to be significantly higher than the current market price of the stock, investors may want to buy shares. If the analyst calculates a lower intrinsic value than the current market price, the stock is considered overvalued and investors may want to sell the stock.

\bigbreak

% \textbf{Technical analysis}
Technical analysis is very different from fundamental analysis. It is an umbrella term of using mathematical method to evaluate stock price. The objective of using technical analysis is not to predict the price, it gives the investors the signal of when to buy or sell a stock by evaluating the statistical trends of stock price Because it is believed that history tends to repeat itself, any oscillation shall finally be reflected in the stock price. Technical analysis often used as a short-term trading strategy. Technical analysis is used generally to evaluate the price changes, but some analysts track numbers of data other than just price, such as trading volume or open interest figures.

\bigbreak

There are hundreds of patterns and signals that a company can generate. Financial researchers have been studying these signals to support technical analysis trading. Various techincal indicators have aslo been develope to help them better understanding the behavior of stock market.

\bigbreak

% Some indicators are focused primarily on identifying the current market trend, including support and resistance areas, while others are focused on determining the strength of a trend and the likelihood of its continuation. Commonly used technical indicators and charting patterns include trendlines, channels, moving averages, and momentum indicators.

% \bigbreak

% \textbf{Why technical indicators}
In this research, we use technical analysis rather than fundamental analysis. The reason is that the data needed for technical analysis are easier to access, which is the stock price. Besides, it is also easier to focuse only on the stock price, not on several statistic data like governance, revenues, earnings, etc.

\bigbreak

% \textbf{How to optimize the parameters of technical indicators}
The technical indicators we use are relative strength index (RSI) and simple moving average (SMA). These two indicators are the most popular amoung the investors. There are several parameters should be given to the indicators before actually using them. These parameters are important because they decide when the investors should enter or exit the stock market.

\bigbreak

% \textbf{Breifly summerize our method}
Finding a good set of parameters becomes a great issue here because there are vast number of combination of the parameters. Finding a good set of parameters to use is a complex problem. Using exhaustive method is probably not a smart idea because it takes to much time and resource. Metaheuristic algorithm is capable of coping this optimization task. There are many metaheuristic algorithms, such as genetic algorithm (GA), ant colony optimization (ACO), and Global-best guided quantum-inspired tabu search algorithm (GNQTS) [], etc. The method that we apply is GNQTS which can find parameters efficently by using global-best as guidance and quantum not gate to escape local optima. Additionally, our research proposed new sliding window of different time frame to maximize the profit. More over, we combine RSI and SMA to form better strategy rather than using single technical indicator.

\bigbreak

In brief, our research applies two technical indicators with GNQTS and sliding windows. By Utilizing GNQTS to optimize parameters for both RSI and SMA and apply them in the U.S. stock market. The results of experiment show that our method is encouraging and is head and shoulder above the buy and hold strategy (B\&H) and traditional strategies.

\end{document}